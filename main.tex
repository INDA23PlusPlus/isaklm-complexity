\documentclass{article}
\usepackage{graphicx} % Required for inserting images
\usepackage{amsmath}
\usepackage[utf8]{inputenc}
\usepackage[swedish]{babel}
\usepackage{parskip}

\usepackage{listings}
\usepackage{color}

\definecolor{dkgreen}{rgb}{0,0.6,0}
\definecolor{gray}{rgb}{0.5,0.5,0.5}
\definecolor{mauve}{rgb}{0.58,0,0.82}

\lstset{frame=tb,
  language=Java,
  aboveskip=3mm,
  belowskip=3mm,
  showstringspaces=false,
  columns=flexible,
  basicstyle={\small\ttfamily},
  numbers=none,
  numberstyle=\tiny\color{gray},
  keywordstyle=\color{blue},
  commentstyle=\color{dkgreen},
  stringstyle=\color{mauve},
  breaklines=true,
  breakatwhitespace=true,
  tabsize=3
}

\title{Korrekthet och komplexitet}
\author{Isak Livner Mäkitalo}
\date{15 November 2023}

\begin{document}

\maketitle

\begin{enumerate}
  \item
  \begin{itemize}
      \item [a)]
      Vill visa att:
      \[VL(n)=HL(n)\]
      Där:
      \[VL(n)=\sum_{i=1}^{n}i^2\]
      \[HL(n)=\frac{n(n+1)(2n+1)}{6}\]
      Basfall \(n=1\):
      \[VL(1)=\sum_{i=1}^{1}i^2=1^2=1\]
      \[HL(1)=\frac{1(1+1)(2+1)}{6}=\frac{6}{6}=1\]
      Induktionsantagande: \(VL(s)=HL(s)\) för något \(s\geq1\) \newline
      Induktionssteg: Visa att \(VL(s+1)=HL(s+1)\) \newline
      \[VL(s+1)=\sum_{i=1}^{s+1}i^2=\sum_{i=1}^{s}i^2+(s+1)^2=VL(s)+(s+1)^2=\]
      \centerline{\{Enligt induktionsantagandet\}}
      \[=HL(s)+(s+1)^2=\frac{s(s+1)(2s+1)}{6}+(s+1)^2=\]
      \[=\frac{s(s+1)(2s+1)}{6}+\frac{6(s+1)(s+1)}{6}=\frac{s(s+1)(2s+1)+6(s+1)(s+1)}{6}=\]
      \[=\frac{(s+1)(s(2s+1)+6(s+1))}{6}=\frac{(s+1)(s(2s+1)+2(2s+1)+2(s+2))}{6}=\]
      \[=\frac{(s+1)((s+2)(2s+1)+2(s+2))}{6}=\frac{(s+1)(s+2)(2s+3)}{6}=\]
      \[=\frac{(s+1)((s+1)+1)(2(s+1)+1)}{6}=HL(s+1)\]
      Med hjälp av induktionssteget har vi då visat att
      \[VL(s)=HL(s) \implies VL(s+1)=HL(s+1)\]
      Eftersom vi vet att \(VL(1)=HL(1)\) innebär då detta att \(VL(n)=HL(n)\) för \(n \geq 1\)
      \vspace*{\baselineskip}
      \item [b)]
      Vill visa att:
      \[VL(n)=HL(n)\]
      Där:
      \[VL(n)=\sum_{j=1}^{n}(2j-1)\]
      \[HL(n)=n^2\]
      Basfall \(n=1\):
      \[VL(1)=\sum_{j=1}^{1}(2j-1)=2-1=1\]
      \[HL(1)=1^2=1\]
      Induktionsantagande: \(VL(s)=HL(s)\) för något \(s\geq1\) \newline
      Induktionssteg: Visa att \(VL(s+1)=HL(s+1)\) \newline
      \[VL(s+1)=\sum_{j=1}^{s+1}(2j-1)=\sum_{j=1}^{s}(2j-1)+2(s+1)-1=VL(s)+2s+1=\]
      \centerline{\{Enligt induktionsantagandet\}}
      \[=HL(s)+2s+1=s^2+2s+1=(s+1)^2=HL(s+1)\]
      Med hjälp av induktionssteget har vi då visat att
      \[VL(s)=HL(s) \implies VL(s+1)=HL(s+1)\]
      Eftersom vi vet att \(VL(1)=HL(1)\) innebär då detta att \(VL(n)=HL(n)\) för \(n \geq 1\)
  \end{itemize}
  \vspace*{\baselineskip}
  \item
  Vi vill bevisa att följande program alltid beräknar \(x^n\) för \(n \geq 0\)
  \begin{lstlisting}
double expIterative(double x, int n) {
    double res = 1.0;

    for (int i = 0; i < n; i++) {
        res *= x;
    }
    return res;
}
\end{lstlisting}
Basfall \(n=0\):

Algoritmen ignorerar loopen och returnerar det initiella värdet av \texttt{res}, d.v.s 1. Eftersom \(x^0=1\) för alla \(x\), så gäller basfallet.

Induktionsantagande: \(expIterative(x, s)=x^s\) för något \(s \geq 0\) \newline
Induktionssteg: Visa att \(expIterative(x, s+1)=x^{s+1}\) \newline

Enligt induktionsantagandet så kommer \texttt{res} efter de \(s\) första iterationerna av loopen att vara lika med \(x^s\). Därefter multipliceras \texttt{res} med \(x\) och programmet returnerar sedan \texttt{res}.
\[x^s \cdot x = x^{s+1}\]
Med hjälp av induktionssteget har vi då visat att
\[expIterative(x, s)=x^s \implies expIterative(x, s+1)=x^{s+1}\]
Eftersom vi vet att \(expIterative(x, 0)=x^0\) innebär då detta att \(expIterative(x, n)=x^n\) för \(n \geq 0\) \newline

Komplexitet: \newline

Eftersom algoritmen endast innehåller en loop av \texttt{n} iterationer så är det lätt att se att algoritmen har \(O(n)\) komplexitet.
\vspace*{\baselineskip}
  \item
    Vi vill bevisa att följande program alltid beräknar \(x^n\) för \(n \geq 0\)
  \begin{lstlisting}
double expRecursive(double x, int n) {
    if (n <= 4) {
        return expIterative(x, n);
    }

    return expRecursive(x, n/2) *
           expRecursive(x, (n + 1)/2);
}
\end{lstlisting}
Basfall \(0 \leq n \leq 4\):

om \(0 \leq n \leq 4\) så kallar algoritmen på \(expIterative(x, n)\) som vi tidigare bevisade var korrekt, så algoritmen måste vara korrekt för \(0 \leq n \leq 4\) \newline

Induktionsantagande: Det finns något \(s \geq 4\) sådant att \(expRecursive(x, k) = x^k\) om \(0 \leq k \leq s\) \newline
Induktionssteg: Visa att \(expRecursive(x, s+1)=x^{s+1}\) \newline

Då \(s \geq 4\) ger algoritmen att
\[expRecursive(x, s+1) = expRecursive(x, d_1) \cdot expRecursive(x, d_2)\]

där
\[d_1 = \lfloor \frac{s+1}{2} \rfloor \]
\[d_2 = \lfloor \frac{(s+1)+1}{2} \rfloor = \lfloor \frac{(s+1)}{2}+\frac{1}{2} \rfloor \]

Om \(s+1\) är jämn har vi att
\(d_1 + d_2 = \frac{s+1}{2} + \frac{s+1}{2} = s+1\) \newline
Om \(s+1\) är udda har vi att
\(d_1 + d_2 = \frac{s}{2} + \frac{s+2}{2} = s+1\) \newline
I båda fallen har vi alltså att \(d_1+d_2=s+1\) \newline

Eftersom \(d_1 \leq s\) och \(d_2 \leq s\) gäller det enligt induktionsantagandet att
\[expRecursive(x, d_1) \cdot expRecursive(x, d_2) = x^{d_1} \cdot x^{d_2}=x^{s+1}\]

Med hjälp av induktionssteget har vi då visat att
\(expRecursive(x, s+1) = x^{s+1}\) om \(expRecursive(x, k) = x^k\) för alla \(0 \leq k \leq s\) där \(s \geq 4\) \newline
Eftersom vi vet att \(expRecursive(x, n)=x^n\) för \(0 \leq n \leq 4\) så gäller det då att \(expRecursive(x, n)=x^n\) för \(n \geq 0\) \newline

Komplexitet: \newline

\(f(n)\) har \(O(1)\) komplexitet vilket innebär att \(d=0\) \newline
Antalet delproblem \(a=2\) \newline
Storleken av varje delproblem ges av \(\frac{n}{2}\) vilket innebär att \(b=2\) \newline

Då \(a > b^d\) så säger mästarsatsen att algoritmen har komplexiteten:

\[O(n^{log_b(a)})=O(n^{log_2(2)})=O(n)\]



\end{enumerate}

\end{document}